\documentclass[12pt]{article}
\usepackage{amsmath,amssymb}
\usepackage{graphicx}

\title{Introduction to \LaTeX \\ (Part 1 : Basics)}
\author{Tathagata Karmakar}
\date{\today}

\begin{document}
\maketitle

\section{Creating itemized lists}
\begin{itemize}
\item Tea
\item Milk
\item Biscuits
\end{itemize}

\section{Inclusion of figures}

\begin{figure}[h]
\centering
\includegraphics[scale=0.15]{gebril}
\caption{Gebril}
\end{figure}

\section{Typesetting : Text}
\begin{itemize}

\item Words are separated by one or more space.
\item Paragraphs are separated by 
one or more blank lines
\item Space in the source file is collapsed in the output :
\begin{itemize}
\item The rain    in Spain falls    mainly on   the plains.
\end{itemize}

\end{itemize}

\section{Typesetting : Quotation marks and symbols}
\begin{itemize}

\item Single quotes : `text'
\item Double quotes : ``text''
\item Some common characters have special meanings in \LaTeX :
\begin{itemize}
\item percent sign \%
\item hash sign \#
\item ampersand \&
\item dollar sign \$
\end{itemize}

\end{itemize}
To include these symbols in plain text add a backslash on front of the symbol. For example : $\backslash \$, \backslash \%, \backslash \&, \backslash \#$

\section{Handeling Errors}
If an error occurs during compilation of \LaTeX \ code : 
\begin{enumerate}

\item \LaTeX \ detect the bug in the code and tries to point it out by specifying line number and the faulty code.
\item Based in the type of the error occured, \LaTeX \ displays an error message which helps the user to understand the type of error in the code. For example : ``Undefined control sqeuence''  if a \LaTeX \ command is misspelled.

\end{enumerate}

\section{Exercise 1 :}
In March \(\mathrm{2006}\), Congress raised that ceiling an additional \(\mathrm{\$0.79}\) trillion to \(\mathrm{\$8.97}\) trillion, which is approximately \(\mathrm{68\%}\) of GDP. As of October \(\mathrm{4}\), \(\mathrm{2008}\), the "Emergency Economic Stabilization Act of \(\mathrm{2008}\)" raised the current debt ceiling of \(\mathrm{\$11.3}\) trillion.


\section{Typesetting Mathematics : Inline Math}

\begin{itemize}

\item Without inline math : \\
Let a and b be distinct positive integers, and let c = a -b + 1.
\item With inline math : \\
Let $a$ and $b$ be distinct positive integers, and let $c=a-
b+1$

\end{itemize}

\section{Typesetting Mathematics : Notation}
\begin{itemize}
\item Superscripts \& Subscripts : $y = c_2x^2 + c_1x + c_0$
\item Grouping Notation : $F_n = F_{n-1} + F_{n-2}$ 
\item Greek letters : $\mu = A e^{{Q}/{RT}} \Omega = \sum_{k = 1}^{n} \omega_{k}$

\end{itemize}

\section{Typesetting Mathematics : Displayed Equations}

To display bigger equations in seperate line, $\backslash \mathrm{begin} \{\mathrm{equation}\}$ and $\backslash \mathrm{end} \{\mathrm{equation}\}$ is used.

The roots of a quadratic equation are given by
\begin{equation}
x = \frac{-b \pm \sqrt{b^2 - 4ac}}{2a}
\end{equation}

where $a$, $b$ and $c$ are coefficient \ldots

\section{Environments}

Environments in \LaTeX \ are used to implement specific typesetting effects. For example : \emph{equation}, \emph{enumarate}, \emph{itemize}.

The $\backslash \mathrm{begin}$ and $\backslash \mathrm{end}$ commands are used to create many different evnironments.

Itemize : 

\begin{itemize}
\item Biscuits
\item Tea
\end{itemize}

Enumerate : 

\begin{enumerate}
\item Biscuits
\item Tea
\end{enumerate}

\section{Packages}
\begin{itemize}
\item \emph{Packages} are libraries of extra commands and environments, used to implement specific typesettings features in the document.

\item A \emph{package} is added to \LaTeX \ by using the command $\backslash$ $\mathrm{usepackage}$ command in the \emph{preamble} section.

\item Example : $\mathsf{amsmath}$ from the \emph{American Mathematical Society}.

\end{itemize}


\section{Examples with $\mathsf{amsmath}$}

\begin{itemize}
\item Use $\mathsf{equation*}$ for unnumbered eqautions.

\begin{equation*}
\Omega = \sum_{k=1}^{n} \omega_k
\end{equation*}

\item $\mathsf{amsmath}$ defines commands for many common mathematical operators.
\begin{itemize}
\item Without $\mathsf{amsmath}$ operator : 
\begin{equation*}
min_{x,y} (1-x)^2 + 100(y-x^2)^2
\end{equation*}
\item With $\mathsf{amsmath}$ operator : 
\begin{equation*}
\min_{x,y} (1-x)^2 + 100(y-x^2)^2
\end{equation*}

\end{itemize}

\item $\backslash \mathrm{operatorname}$ command for other operators : 
\begin{equation*}
\beta_i = 
\frac{\operatorname{Cov}(R_i, R_m)}
{\operatorname{Var}(R_m)}
\end{equation*}

\item Align a sequence of equations using $\mathsf{align*}$ environment.

\begin{align*}
(x+1)^3 &= (x+1)(x+1)(x+1) \\
		  &= (x+1)(x^2 + 2x + 1) \\
        &= x^3 + 3x^2 + 3x + 1
\end{align*}

Double backslash $\backslash \backslash$ for new line. An ampersand \& to separate the left column from the right column of the equaiton.

\end{itemize}


\section{Exercise : 2}

Let $\mathrm{X_1,X_2,\ldots, X_n}$ be a sequence of independent and identically distributed random variables with $\mathrm{E[X_i] = \mu}$ and $\mathrm{Var]X_i = \sigma^2 < \infty},$ and let

\begin{equation*}
\mathrm{S_n = \frac{1}{n} \sum_{i = 1}^{n} X_i}
\end{equation*}

denote their mean. Then as $n$ approches infinity, the random variables $\mathrm{\sqrt{n}(S_n - \mu)}$ converge in distribution to a normal $\mathrm{N(0,\sigma^2)}$



\end{document}

